\documentclass{article}
\usepackage[utf8]{inputenc}

\title{Create HTML output using GitHub Actions}
\author{Michal Hoftich}
\date{December 2019}
\usepackage{hyperref}
\usepackage{listings}
\newcommand{\cmdname}[1]{\texttt{#1}}
\newcommand\footurl[1]{\footnote{\url{#1}}}
\usepackage{upquote}

\begin{document}

\maketitle

\section{Introduction}

I've set up Docker image\footurl{https://hub.docker.com/repository/docker/michalh21/make4ht-docker/general} for \TeX4ht, \LaTeX\ to XML converter. Thanks to GitHub 
Actions\footurl{https://help.github.com/en/actions/automating-your-workflow-with-github-actions/about-github-actions}, it is possible to use it for Overleaf document to HTML conversion.

\section{Setup}

\subsection{On Overleaf}

First step is to sync your Overleaf project with your GitHub account, following a guide at 
Overleaf\footnote{\href{https://www.overleaf.com/learn/how-to/How_do_I_connect_an_Overleaf_project_with_a_repo_on_GitHub,_GitLab_or_BitBucket\%3F}
{\texttt{https://www.overleaf.com/learn/how-to/How_do_I_connect_an_Overleaf_project_with_a_repo_on_GitHub,_GitLab_or_BitBucket\%3F}}}. 
Don't forget to run the \cmdname{Sync -> GitHub} command from Overleaf main menu every time you updated the document.

\subsection{On GitHub}
Next step is to configure actions in the Github project created for your document. Two steps are necessary -- first  one compiles the document to HTML using \TeX4ht, the second step publishes the generated HTML files on the Web using GitHub Pages\footurl{https://pages.github.com/}

For the web publishing we will use the \verb|actions-gh-pages| action\footurl{https://github.com/peaceiris/actions-gh-pages}. Follow \href{https://github.com/peaceiris/actions-gh-pages\#1-add-ssh-deploy-key}{the instructions} and create the keys.

When the keys are set up, you can create the workflow file. Select the \cmdname{Actions} tab and click the \cmdname{Set up a workflow yourself} button. It will open an editor with a YAML file for the Action workflow. Replace it with the following content:

\begin{lstlisting}{yaml}
name: CI
on: [push]
jobs:
  build:
    runs-on: ubuntu-latest
    steps:
    - uses: actions/checkout@v1
    - name: Run make4ht
      uses: docker://michalh21/make4ht-docker:latest
      env:
        command: "make4ht -u -d out main.tex"
    - name: Publish the web pages
      uses: peaceiris/actions-gh-pages@v2.5.0
      env:
        ACTIONS_DEPLOY_KEY: ${{ secrets.ACTIONS_DEPLOY_KEY }}
        PUBLISH_BRANCH: gh-pages
        PUBLISH_DIR: ./out    
\end{lstlisting}

The important part of the configuration is the \cmdname{command} key. It contains the actual command used for the compilation. We use Make4ht\footurl{https://ctan.org/pkg/make4ht?lang=en}, build system for \TeX4ht.
Command \verb|"make4ht -u -d out main.tex"| creates UTF-8 encoded HTML file from the \cmdname{main.tex} input file. The \texttt{-d} option specifies the output directory for the HTML files. This directory will be used for the web publishing and should be passed to the \verb|PUBLISH_DIR|.

The web will be published at \url{https://yourgitubusename.github.io/project_name/main.html}. 
\subsection{Links}
\begin{itemize}
    \item This document: \url{https://www.kodymirus.cz/overleaf-html-sample/main.html}
    \item Source repository: \url{https://github.com/michal-h21/overleaf-html-sample} 
\end{itemize}
 



\end{document}
